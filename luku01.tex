\chapter{Introduction}

Competitive programming combines two topics:
(1) the design of algorithms and (2) the implementation of algorithms.

The \key{design of algorithms} consists of problem solving
and mathematical thinking.
Skills for analyzing problems and solving them
creatively are needed.
An algorithm for solving a problem
has to be both correct and efficient,
and the core of the problem is often
how to invent an efficient algorithm.

Theoretical knowledge of algorithms
is very important to competitive programmers.
Typically, a solution to a problem is
a combination of well-known techniques and
new insights.
The techniques that appear in competitive programming
also form the basis for the scientific research
of algorithms.

The \key{implementation of algorithms} requires good
programming skills.
In competitive programming, the solutions
are graded by testing an implemented algorithm
using a set of test cases.
Thus, it is not enough that the idea of the
algorithm is correct, but the implementation has
also to be correct.

Good coding style in contests is
straightforward and concise.
The programs should be written quickly,
because there is not much time available.
Unlike in traditional software engineering,
the programs are short (usually at most some
hundreds of lines) and it is not needed to 
maintain them after the contest.

\section{Programming languages}

\index{programming language}

At the moment, the most popular programming
languages in contests are C++, Python and Java.
For example, in Google Code Jam 2016,
among the best 3,000 participants,
73 \% used C++,
15 \% used Python and
10 \% used Java \cite{goo16}.
Some participants also used several languages.

Many people think that C++ is the best choice
for a competitive programmer,
and C++ is nearly always available in
contest systems.
The benefits in using C++ are that
it is a very efficient language and
its standard library contains a 
large collection
of data structures and algorithms.

On the other hand, it is good to
master several languages and know
their strengths.
For example, if large integers are needed
in the problem,
Python can be a good choice, because it
contains built-in operations for
calculating with large integers.
Still, most problems in programming contests
are set so that
using a specific programming language
is not an unfair advantage.

All example programs in this book are written in C++,
and the standard library's
data structures and algorithms are often used.
The programs follow the C++11 standard,
that can be used in most contests nowadays.
If you cannot program in C++ yet,
now it is a good time to start learning.

\subsubsection{C++ template}

A typical C++ template for competitive programming
looks like this:

\begin{lstlisting}
#include <bits/stdc++.h>

using namespace std;

int main() {
    // solution comes here
}
\end{lstlisting}

The \texttt{\#include} line at the beginning
of the code is a feature of the \texttt{g++} compiler
that allows us to include the entire standard library.
Thus, it is not needed to separately include
libraries such as \texttt{iostream},
\texttt{vector} and \texttt{algorithm},
but they are available automatically.

The \texttt{using} line determines
that the classes and functions
of the standard library can be used directly
in the code.
Without the \texttt{using} line we should write,
for example, \texttt{std::cout},
but now it suffices to write \texttt{cout}.

The code can be compiled using the following command:

\begin{lstlisting}
g++ -std=c++11 -O2 -Wall code.cpp -o code
\end{lstlisting}

This command produces a binary file \texttt{code}
from the source code \texttt{code.cpp}.
The compiler follows the C++11 standard
(\texttt{-std=c++11}),
optimizes the code (\texttt{-O2})
and shows warnings about possible errors (\texttt{-Wall}).

\section{Input and output}

\index{input and output}

In most contests, standard streams are used for
reading input and writing output.
In C++, the standard streams are
\texttt{cin} for input and \texttt{cout} for output.
In addition, the C functions
\texttt{scanf} and \texttt{printf} can be used.

The input for the program usually consists of
numbers and strings that are separated with
spaces and newlines.
They can be read from the \texttt{cin} stream
as follows:

\begin{lstlisting}
int a, b;
string x;
cin >> a >> b >> x;
\end{lstlisting}

This kind of code always works,
assuming that there is at least one space
or newline between each element in the input.
For example, the above code can read
both the following inputs:
\begin{lstlisting}
123 456 monkey
\end{lstlisting}
\begin{lstlisting}
123    456
monkey
\end{lstlisting}
The \texttt{cout} stream is used for output
as follows:
\begin{lstlisting}
int a = 123, b = 456;
string x = "monkey";
cout << a << " " << b << " " << x << "\n";
\end{lstlisting}

Input and output is sometimes
a bottleneck in the program.
The following lines at the beginning of the code
make input and output more efficient:

\begin{lstlisting}
ios_base::sync_with_stdio(0);
cin.tie(0);
\end{lstlisting}

Note that the newline \texttt{"\textbackslash n"}
works faster than \texttt{endl},
because \texttt{endl} always causes
a flush operation.

The C functions \texttt{scanf}
and \texttt{printf} are an alternative
to the C++ standard streams.
They are usually a bit faster,
but they are also more difficult to use.
The following code reads two integers from the input:
\begin{lstlisting}
int a, b;
scanf("%d %d", &a, &b);
\end{lstlisting}
The following code prints two integers:
\begin{lstlisting}
int a = 123, b = 456;
printf("%d %d\n", a, b);
\end{lstlisting}

Sometimes the program should read a whole line
from the input, possibly with spaces.
This can be accomplished by using the
\texttt{getline} function:

\begin{lstlisting}
string s;
getline(cin, s);
\end{lstlisting}

If the amount of data is unknown, the following
loop is useful:
\begin{lstlisting}
while (cin >> x) {
    // code
}
\end{lstlisting}
This loop reads elements from the input
one after another, until there is no
more data available in the input.

In some contest systems, files are used for
input and output.
An easy solution for this is to write
the code as usual using standard streams,
but add the following lines to the beginning of the code:
\begin{lstlisting}
freopen("input.txt", "r", stdin);
freopen("output.txt", "w", stdout);
\end{lstlisting}
After this, the program reads the input from the file
''input.txt'' and writes the output to the file
''output.txt''.

\section{Working with numbers}

\index{integer}

\subsubsection{Integers}

The most used integer type in competitive programming
is \texttt{int}, that is a 32-bit type with
value range $-2^{31} \ldots 2^{31}-1$
or about $-2 \cdot 10^9 \ldots 2 \cdot 10^9$.
If the type \texttt{int} is not enough,
the 64-bit type \texttt{long long} can be used,
with value range $-2^{63} \ldots 2^{63}-1$
or about $-9 \cdot 10^{18} \ldots 9 \cdot 10^{18}$.

The following code defines a
\texttt{long long} variable:
\begin{lstlisting}
long long x = 123456789123456789LL;
\end{lstlisting}
The suffix \texttt{LL} means that the
type of the number is \texttt{long long}.

A common error when using the type \texttt{long long}
is that the type \texttt{int} is still used somewhere
in the code.
For example, the following code contains
a subtle error:

\begin{lstlisting}
int a = 123456789;
long long b = a*a;
cout << b << "\n"; // -1757895751
\end{lstlisting}

Even though the variable \texttt{b} is of type \texttt{long long},
both numbers in the expression \texttt{a*a}
are of type \texttt{int} and the result is
also of type \texttt{int}.
Because of this, the variable \texttt{b} will
contain a wrong result.
The problem can be solved by changing the type
of \texttt{a} to \texttt{long long} or
by changing the expression to \texttt{(long long)a*a}.

Usually contest problems are set so that the
type \texttt{long long} is enough.
Still, it is good to know that
the \texttt{g++} compiler also provides
an 128-bit type \texttt{\_\_int128\_t}
with value range
$-2^{127} \ldots 2^{127}-1$ or $-10^{38} \ldots 10^{38}$.
However, this type is not available in all contest systems.

\subsubsection{Modular arithmetic}

\index{remainder}
\index{modular arithmetic}

We denote by $x \bmod m$ the remainder
when $x$ is divided by $m$.
For example, $17 \bmod 5 = 2$,
because $17 = 3 \cdot 5 + 2$.

Sometimes, the answer to a problem is a
very large number but it is enough to
output it ''modulo $m$'', i.e.,
the remainder when the answer is divided by $m$
(for example, ''modulo $10^9+7$'').
The idea is that even if the actual answer
may be very large,
it suffices to use the types
\texttt{int} and \texttt{long long}.

An important property of the remainder is that
in addition, subtraction and multiplication,
the remainder can be taken before the operation:

\[
\begin{array}{rcr}
(a+b) \bmod m & = & (a \bmod m + b \bmod m) \bmod m \\
(a-b) \bmod m & = & (a \bmod m - b \bmod m) \bmod m \\
(a \cdot b) \bmod m & = & (a \bmod m \cdot b \bmod m) \bmod m
\end{array}
\]

Thus, we can take the remainder after every operation
and the numbers will never become too large.

For example, the following code calculates $n!$,
the factorial of $n$, modulo $m$:
\begin{lstlisting}
long long x = 1;
for (int i = 2; i <= n i++) {
    x = (x*i)%m;
}
cout << x << "\n";
\end{lstlisting}

Usually the remainder should be always
be between $0\ldots m-1$.
However, in C++ and other languages,
the remainder of a negative number
is either zero or negative.
An easy way to make sure there
are no negative remainders is to first calculate
the remainder as usual and then add $m$
if the result is negative:
\begin{lstlisting}
x = x%m;
if (x < 0) x += m;
\end{lstlisting}
However, this is only needed when there
are subtractions in the code and the
remainder may become negative.

\subsubsection{Floating point numbers}

\index{floating point number}

The usual floating point types in
competitive programming are
the 64-bit \texttt{double}
and, as an extension in the \texttt{g++} compiler,
the 80-bit \texttt{long double}.
In most cases, \texttt{double} is enough,
but \texttt{long double} is more accurate.

The required precision of the answer
is usually given in the problem statement.
An easy way to output the answer is to use
the \texttt{printf} function
and give the number of decimal places
in the formatting string.
For example, the following code prints
the value of $x$ with 9 decimal places:

\begin{lstlisting}
printf("%.9f\n", x);
\end{lstlisting}

A difficulty when using floating point numbers
is that some numbers cannot be represented
accurately as floating point numbers,
but there will be rounding errors.
For example, the result of the following code
is surprising:

\begin{lstlisting}
double x = 0.3*3+0.1;
printf("%.20f\n", x); // 0.99999999999999988898
\end{lstlisting}

Due to a rounding error,
the value of \texttt{x} is a bit smaller than 1,
while the correct value would be 1.

It is risky to compare floating point numbers
with the \texttt{==} operator,
because it is possible that the values should
be equal but they are not because of rounding.
A better way to compare floating point numbers
is to assume that two numbers are equal
if the difference between them is $\varepsilon$,
where $\varepsilon$ is a small number.

In practice, the numbers can be compared
as follows ($\varepsilon=10^{-9}$):

\begin{lstlisting}
if (abs(a-b) < 1e-9) {
    // a and b are equal
}
\end{lstlisting}

Note that while floating point numbers are inaccurate,
integers up to a certain limit can be still
represented accurately.
For example, using \texttt{double},
it is possible to accurately represent all
integers whose absolute value is at most $2^{53}$.

\section{Shortening code}

Short code is ideal in competitive programming,
because programs should be written
as fast as possible.
Because of this, competitive programmers often define
shorter names for datatypes and other parts of code.

\subsubsection{Type names}
\index{tuppdef@\texttt{typedef}}
Using the command \texttt{typedef}
it is possible to give a shorter name
to a datatype.
For example, the name \texttt{long long} is long,
so we can define a shorter name \texttt{ll}:
\begin{lstlisting}
typedef long long ll;
\end{lstlisting}
After this, the code
\begin{lstlisting}
long long a = 123456789;
long long b = 987654321;
cout << a*b << "\n";
\end{lstlisting}
can be shortened as follows:
\begin{lstlisting}
ll a = 123456789;
ll b = 987654321;
cout << a*b << "\n";
\end{lstlisting}

The command \texttt{typedef}
can also be used with more complex types.
For example, the following code gives
the name \texttt{vi} for a vector of integers
and the name \texttt{pi} for a pair
that contains two integers.
\begin{lstlisting}
typedef vector<int> vi;
typedef pair<int,int> pi;
\end{lstlisting}

\subsubsection{Macros}
\index{macro}
Another way to shorten the code is to define
\key{macros}.
A macro means that certain strings in
the code will be changed before the compilation.
In C++, macros are defined using the
command \texttt{\#define}.

For example, we can define the following macros:
\begin{lstlisting}
#define F first
#define S second
#define PB push_back
#define MP make_pair
\end{lstlisting}
After this, the code
\begin{lstlisting}
v.push_back(make_pair(y1,x1));
v.push_back(make_pair(y2,x2));
int d = v[i].first+v[i].second;
\end{lstlisting}
can be shortened as follows:
\begin{lstlisting}
v.PB(MP(y1,x1));
v.PB(MP(y2,x2));
int d = v[i].F+v[i].S;
\end{lstlisting}

A macro can also have parameters
which makes it possible to shorten loops and other
structures.
For example, we can define the following macro:
\begin{lstlisting}
#define REP(i,a,b) for (int i = a; i <= b; i++)
\end{lstlisting}
After this, the code
\begin{lstlisting}
for (int i = 1; i <= n; i++) {
    search(i);
}
\end{lstlisting}
can be shortened as follows:
\begin{lstlisting}
REP(i,1,n) {
    search(i);
}
\end{lstlisting}

\section{Mathematics}

Mathematics plays an important role in competitive
programming, and it is not possible to become
a successful competitive programmer without
having good mathematical skills.
This section discusses some important
mathematical concepts and formulas that
are needed later in the book.

\subsubsection{Sum formulas}

Each sum of the form
\[\sum_{x=1}^n x^k = 1^k+2^k+3^k+\ldots+n^k,\]
where $k$ is a positive integer,
has a closed-form formula that is a
polynomial of degree $k+1$.
For example,
\[\sum_{x=1}^n x = 1+2+3+\ldots+n = \frac{n(n+1)}{2}\]
and
\[\sum_{x=1}^n x^2 = 1^2+2^2+3^2+\ldots+n^2 = \frac{n(n+1)(2n+1)}{6}.\]

An \key{arithmetic progression} is a \index{arithmetic progression}
sequence of numbers
where the difference between any two consecutive
numbers is constant.
For example,
\[3, 7, 11, 15\]
is an arithmetic progression with constant 4.
The sum of an arithmetic progression can be calculated
using the formula
\[\frac{n(a+b)}{2}\]
where $a$ is the first number,
$b$ is the last number and
$n$ is the amount of numbers.
For example,
\[3+7+11+15=\frac{4 \cdot (3+15)}{2} = 36.\]
The formula is based on the fact
that the sum consists of $n$ numbers and
the value of each number is $(a+b)/2$ on average.

\index{geometric progression}
A \key{geometric progression} is a sequence
of numbers
where the ratio between any two consecutive
numbers is constant.
For example,
\[3,6,12,24\]
is a geometric progression with constant 2.
The sum of a geometric progression can be calculated
using the formula
\[\frac{bx-a}{x-1}\]
where $a$ is the first number,
$b$ is the last number and the
ratio between consecutive numbers is $x$.
For example,
\[3+6+12+24=\frac{24 \cdot 2 - 3}{2-1} = 45.\]

This formula can be derived as follows. Let
\[ S = a + ax + ax^2 + \cdots + b .\]
By multiplying both sides by $x$, we get
\[ xS = ax + ax^2 + ax^3 + \cdots + bx,\]
and solving the equation
\[ xS-S = bx-a\]
yields the formula.

A special case of a sum of a geometric progression is the formula
\[1+2+4+8+\ldots+2^{n-1}=2^n-1.\]

\index{harmonic sum}

A \key{harmonic sum} is a sum of the form
\[ \sum_{x=1}^n \frac{1}{x} = 1+\frac{1}{2}+\frac{1}{3}+\ldots+\frac{1}{n}.\]

An upper bound for a harmonic sum is $\log_2(n)+1$.
Namely, we can
modify each term $1/k$ so that $k$ becomes
the nearest power of two that does not exceed $k$.
For example, when $n=6$, we can estimate
the sum as follows:
\[ 1+\frac{1}{2}+\frac{1}{3}+\frac{1}{4}+\frac{1}{5}+\frac{1}{6} \le
1+\frac{1}{2}+\frac{1}{2}+\frac{1}{4}+\frac{1}{4}+\frac{1}{4}.\]
This upper bound consists of $\log_2(n)+1$ parts
($1$, $2 \cdot 1/2$, $4 \cdot 1/4$, etc.),
and the value of each part is at most 1.

\subsubsection{Set theory}

\index{set theory}
\index{set}
\index{intersection}
\index{union}
\index{difference}
\index{subset}
\index{universal set}
\index{complement}

A \key{set} is a collection of elements.
For example, the set
\[X=\{2,4,7\}\]
contains elements 2, 4 and 7.
The symbol $\emptyset$ denotes an empty set,
and $|S|$ denotes the size of a set $S$,
i.e., the number of elements in the set.
For example, in the above set, $|X|=3$.

If a set $S$ contains an element $x$,
we write $x \in S$,
and otherwise we write $x \notin S$.
For example, in the above set
\[4 \in X \hspace{10px}\textrm{and}\hspace{10px} 5 \notin X.\]

\begin{samepage}
New sets can be constructed using set operations:
\begin{itemize}
\item The \key{intersection} $A \cap B$ consists of elements
that are both in $A$ and $B$.
For example, if $A=\{1,2,5\}$ and $B=\{2,4\}$,
then $A \cap B = \{2\}$.
\item The \key{union} $A \cup B$ consists of elements
that are in $A$ or $B$ or both.
For example, if $A=\{3,7\}$ and $B=\{2,3,8\}$,
then $A \cup B = \{2,3,7,8\}$.
\item The \key{complement} $\bar A$ consists of elements
that are not in $A$.
The interpretation of a complement depends on
the \key{universal set} that contains all possible elements.
For example, if $A=\{1,2,5,7\}$ and the universal set is
$\{1,2,\ldots,10\}$, then $\bar A = \{3,4,6,8,9,10\}$.
\item The \key{difference} $A \setminus B = A \cap \bar B$
consists of elements that are in $A$ but not in $B$.
Note that $B$ can contain elements that are not in $A$.
For example, if $A=\{2,3,7,8\}$ and $B=\{3,5,8\}$,
then $A \setminus B = \{2,7\}$.
\end{itemize}
\end{samepage}

If each element of $A$ also belongs to $S$,
we say that $A$ is a \key{subset} of $S$,
denoted by $A \subset S$.
A set $S$ always has $2^{|S|}$ subsets,
including the empty set.
For example, the subsets of the set $\{2,4,7\}$ are
\begin{center}
$\emptyset$,
$\{2\}$, $\{4\}$, $\{7\}$, $\{2,4\}$, $\{2,7\}$, $\{4,7\}$ and $\{2,4,7\}$.
\end{center}

Often used sets are
$\mathbb{N}$ (natural numbers),
$\mathbb{Z}$ (integers),
$\mathbb{Q}$ (rational numbers) and
$\mathbb{R}$ (real numbers).
The set $\mathbb{N}$
can be defined in two ways, depending
on the situation:
either $\mathbb{N}=\{0,1,2,\ldots\}$
or $\mathbb{N}=\{1,2,3,...\}$.

We can also construct a set using a rule of the form
\[\{f(n) : n \in S\},\]
where $f(n)$ is some function.
This set contains all elements of the form $f(n)$,
where $n$ is an element in $S$.
For example, the set
\[X=\{2n : n \in \mathbb{Z}\}\]
contains all even integers.

\subsubsection{Logic}

\index{logic}
\index{negation}
\index{conjuction}
\index{disjunction}
\index{implication}
\index{equivalence}

The value of a logical expression is either
\key{true} (1) or \key{false} (0).
The most important logical operators are
$\lnot$ (\key{negation}),
$\land$ (\key{conjunction}),
$\lor$ (\key{disjunction}),
$\Rightarrow$ (\key{implication}) and
$\Leftrightarrow$ (\key{equivalence}).
The following table shows the meaning of these operators:

\begin{center}
\begin{tabular}{rr|rrrrrrr}
$A$ & $B$ & $\lnot A$ & $\lnot B$ & $A \land B$ & $A \lor B$ & $A \Rightarrow B$ & $A \Leftrightarrow B$ \\
\hline
0 & 0 & 1 & 1 & 0 & 0 & 1 & 1 \\
0 & 1 & 1 & 0 & 0 & 1 & 1 & 0 \\
1 & 0 & 0 & 1 & 0 & 1 & 0 & 0 \\
1 & 1 & 0 & 0 & 1 & 1 & 1 & 1 \\
\end{tabular}
\end{center}

The expression $\lnot A$ has the opposite value of $A$.
The expression $A \land B$ is true if both $A$ and $B$
are true,
and the expression $A \lor B$ is true if $A$ or $B$ or both
are true.
The expression $A \Rightarrow B$ is true
if whenever $A$ is true, also $B$ is true.
The expression $A \Leftrightarrow B$ is true
if $A$ and $B$ are both true or both false.

\index{predicate}

A \key{predicate} is an expression that is true or false
depending on its parameters.
Predicates are usually denoted by capital letters.
For example, we can define a predicate $P(x)$
that is true exactly when $x$ is a prime number.
Using this definition, $P(7)$ is true but $P(8)$ is false.

\index{quantifier}

A \key{quantifier} connects a logical expression
to the elements of a set.
The most important quantifiers are
$\forall$ (\key{for all}) and $\exists$ (\key{there is}).
For example,
\[\forall x (\exists y (y < x))\]
means that for each element $x$ in the set,
there is an element $y$ in the set
such that $y$ is smaller than $x$.
This is true in the set of integers,
but false in the set of natural numbers.

Using the notation described above,
we can express many kinds of logical propositions.
For example,
\[\forall x ((x>1 \land \lnot P(x)) \Rightarrow (\exists a (\exists b (x = ab \land a > 1 \land b > 1))))\]
means that if a number $x$ is larger than 1
and not a prime number,
then there are numbers $a$ and $b$
that are larger than $1$ and whose product is $x$.
This proposition is true in the set of integers.

\subsubsection{Functions}

The function $\lfloor x \rfloor$ rounds the number $x$
down to an integer, and the function
$\lceil x \rceil$ rounds the number $x$
up to an integer. For example,
\[ \lfloor 3/2 \rfloor = 1 \hspace{10px} \textrm{and} \hspace{10px} \lceil 3/2 \rceil = 2.\]

The functions $\min(x_1,x_2,\ldots,x_n)$
and $\max(x_1,x_2,\ldots,x_n)$
give the smallest and largest of values
$x_1,x_2,\ldots,x_n$.
For example,
\[ \min(1,2,3)=1 \hspace{10px} \textrm{and} \hspace{10px} \max(1,2,3)=3.\]

\index{factorial}

The \key{factorial} $n!$ can be defined
\[\prod_{x=1}^n x = 1 \cdot 2 \cdot 3 \cdot \ldots \cdot n\]
or recursively
\[
\begin{array}{lcl}
0! & = & 1 \\
n! & = & n \cdot (n-1)! \\
\end{array}
\]

\index{Fibonacci number}

The \key{Fibonacci numbers} arise in many situations.
They can be defined recursively as follows:
\[
\begin{array}{lcl}
f(0) & = & 0 \\
f(1) & = & 1 \\
f(n) & = & f(n-1)+f(n-2) \\
\end{array}
\]
The first Fibonacci numbers are
\[0, 1, 1, 2, 3, 5, 8, 13, 21, 34, 55, \ldots\]
There is also a closed-form formula
for calculating Fibonacci numbers:
\[f(n)=\frac{(1 + \sqrt{5})^n - (1-\sqrt{5})^n}{2^n \sqrt{5}}.\]

\subsubsection{Logarithms}

\index{logarithm}

The \key{logarithm} of a number $x$
is denoted $\log_k(x)$, where $k$ is the base
of the logarithm.
According to the definition,
$\log_k(x)=a$ exactly when $k^a=x$.

A useful property of logarithms is
that $\log_k(x)$ equals the number of times
we have to divide $x$ by $k$ before we reach 
the number 1.
For example, $\log_2(32)=5$
because 5 divisions are needed:

\[32 \rightarrow 16 \rightarrow 8 \rightarrow 4 \rightarrow 2 \rightarrow 1 \]

Logarithms are often used in the analysis of
algorithms, because many efficient algorithms
halve something at each step.
Hence, we can estimate the efficiency of such algorithms
using logarithms.

The logarithm of a product is
\[\log_k(ab) = \log_k(a)+\log_k(b),\]
and consequently,
\[\log_k(x^n) = n \cdot \log_k(x).\]
In addition, the logarithm of a quotient is
\[\log_k\Big(\frac{a}{b}\Big) = \log_k(a)-\log_k(b).\]
Another useful formula is
\[\log_u(x) = \frac{\log_k(x)}{\log_k(u)},\]
and using this, it is possible to calculate
logarithms to any base if there is a way to
calculate logarithms to some fixed base.

\index{natural logarithm}

The \key{natural logarithm} $\ln(x)$ of a number $x$
is a logarithm whose base is $e \approx 2{,}71828$.

Another property of logarithms is that
the number of digits of an integer $x$ in base $b$ is
$\lfloor \log_b(x)+1 \rfloor$.
For example, the representation of
$123$ in base $2$ is 1111011 and
$\lfloor \log_2(123)+1 \rfloor = 7$.

