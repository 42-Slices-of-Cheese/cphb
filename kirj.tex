\begin{thebibliography}{9}

\bibitem{aho83}
  A. V. Aho, J. E. Hopcroft and J. Ullman.
  \emph{Data Structures and Algorithms},
  Addison-Wesley, 1983.

\bibitem{asp79}
  B. Aspvall, M. F. Plass and R. E. Tarjan.
  A linear-time algorithm for testing the truth of certain quantified boolean formulas.
  \emph{Information Processing Letters}, 8(3):121--123, 1979.

\bibitem{bel58}
  R. Bellman.
  On a routing problem.
  \emph{Quarterly of Applied Mathematics}, 16(1):87--90, 1958.

\bibitem{ben86}
  J. Bentley.
  \emph{Programming Pearls}.
  Addison-Wesley, 1986.

\bibitem{ber93}
  O. Berkman and U. Vishkin.
  Recursive star-tree parallel data structure.
  \emph{SIAM Journal on Computing},
  22(2):221--242, 1993.

\bibitem{cod15}
  Codeforces: On ''Mo's algorithm'',
  \url{http://codeforces.com/blog/entry/20032}

\bibitem{dij59}
  E. W. Dijkstra.
  A note on two problems in connexion with graphs.
  \emph{Numerische Mathematik}, 1(1):269--271, 1959.

\bibitem{edm65}
  J. Edmonds.
  Paths, trees, and flowers.
  \emph{Canadian Journal of Mathematics}, 17(3):449--467, 1965.

\bibitem{edm72}
  J. Edmonds and R. M. Karp.
  Theoretical improvements in algorithmic efficiency for network flow problems.
  \emph{Journal of the ACM}, 19(2):248--264, 1972.

\bibitem{fan94}
  D. Fanding.
  A faster algorithm for shortest-path -- SPFA.
  \emph{Journal of Southwest Jiaotong University}, 2, 1994.

\bibitem{fen94}
  P. M. Fenwick.
  A new data structure for cumulative frequency tables.
  \emph{Software: Practice and Experience}, 24(3):327--336, 1994.

\bibitem{fis11}
  J. Fischer and V. Heun.
  Space-efficient preprocessing schemes for range minimum queries on static arrays.
  \emph{SIAM Journal on Computing}, 40(2):465--492, 2011.

\bibitem{flo62}
  R. W. Floyd
  Algorithm 97: shortest path.
  \emph{Communications of the ACM}, 5(6):345, 1962.

\bibitem{for56}
  L. R. Ford and D. R. Fulkerson.
  Maximal flow through a network.
  \emph{Canadian Journal of Mathematics}, 8(3):399--404, 1956.

\bibitem{gal14}
  F. Le Gall.
  Powers of tensors and fast matrix multiplication.
  In \emph{Proceedings of the 39th International Symposium on Symbolic and Algebraic Computation},
  296--303.

\bibitem{gar79}
  M. R. Garey and D. S. Johnson.
  \emph{Computers and Intractability:
  A Guide to the Theory of NP-Completeness},
  W. H. Freeman and Company, 1979.

\bibitem{goo16}
  Google Code Jam Statistics (2016),
  \url{https://www.go-hero.net/jam/16}

\bibitem{gro14}
  A. Grønlund and S. Pettie.
  Threesomes, degenerates, and love triangles.
  \emph{2014 IEEE 55th Annual Symposium on Foundations of Computer Science},
  621--630, 2014.

\bibitem{gus97}
  D. Gusfield.
  \emph{Algorithms on Strings, Trees and Sequences:
  Computer Science and Computational Biology},
  Cambridge University Press, 1997.

\bibitem{huf52}
  A method for the construction of minimum-redundancy codes.
  \emph{Proceedings of the IRE}, 40(9):1098--1101, 1952.

\bibitem{kas61}
  P. W. Kasteleyn.  
  The statistics of dimers on a lattice: I. The number of dimer arrangements on a quadratic lattice.
  \emph{Physica}, 27(12):1209--1225, 1961.

\bibitem{ken06}
  C. Kent, G. m. Landau and M. Ziv-Ukelson.
  On the complexity of sparse exon assembly.
  \emph{Journal of Computational Biology}, 13(5):1013--1027, 2006.

\bibitem{kru56}
  J. B. Kruskal.
  On the shortest spanning subtree of a graph and the traveling salesman problem.
  \emph{Proceedings of the American Mathematical Society}, 7(1):48--50, 1956.

\bibitem{pac13}
  J. Pachocki and J. Radoszweski.
  Where to use and how not to use polynomial string hashing.
  \emph{Olympiads in Informatics}, 2013.

\bibitem{pri57}
  R. C. Prim.
  Shortest connection networks and some generalizations.
  \emph{Bell System Technical Journal}, 36(6):1389--1401, 1957.

\bibitem{sha81}
  M. Sharir.
  A strong-connectivity algorithm and its applications in data flow analysis.
  \emph{Computers \& Mathematics with Applications}, 7(1):67--72, 1981.

\bibitem{str69}
  V. Strassen.
  Gaussian elimination is not optimal.
  \emph{Numerische Mathematik}, 13(4):354--356, 1969.

\bibitem{tar84}
  R. E. Tarjan and U. Vishkin.
  Finding biconnected components and computing tree functions
  in logarithmic parallel time.
  \emph{25th Annual Symposium on Foundations of Computer Science}, 12--20, 1984.

\bibitem{tem61}
  H. N. V. Temperley and M. E. Fisher.
  Dimer problem in statistical mechanics -- an exact result.
  \emph{Philosophical Magazine}, 6(68):1061--1063, 1961.

\end{thebibliography}
% 
% 
% \chapter*{Literature}
% \markboth{\MakeUppercase{Literature}}{}
% \addcontentsline{toc}{chapter}{Literature}
% 
% \subsubsection{Textbooks on algorithms}
% 
% \begin{itemize}
% \item T. H. Cormen, C. E. Leiserson,
% R. L Rivest and C. Stein:
% \emph{Introduction to Algorithms},
% MIT Press, 2009 (3rd edition)
% \item J. Kleinberg and É. Tardos:
% \emph{Algorithm Design},
% Pearson, 2005
% \item S. S. Skiena:
% \emph{The Algorithm Design Manual},
% Springer, 2008 (2nd edition)
% \end{itemize}
% 
% \subsubsection{Textbooks on competitive programming}
% 
% \begin{itemize}
% \item K. Diks, T. Idziaszek,
% J. Łącki and J. Radoszewski:
% \emph{Looking for a Challenge?},
% University of Warsaw, 2012
% \item S. Halim and F. Halim:
% \emph{Competitive Programming},
% 2013 (3rd edition)
% \item S. S. Skiena and M. A. Revilla:
% \emph{Programming Challenges: The Programming
% Contest Training Manual},
% Springer, 2003
% \end{itemize}
% 
% \subsubsection{Other books}
% 
% \begin{itemize}
% \item J. Bentley:
% \emph{Programming Pearls},
% Addison-Wesley, 1999 (2nd edition)
% \end{itemize}